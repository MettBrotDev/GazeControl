\chapter{Introduction}\label{chapter:introduction}
%
Human vision is inherently selective. Rather than processing the entire visual field uniformly, 
the human eye relies on a foveated view, where high-resolution vision is concentrated in a small central region (the fovea), 
while the surrounding peripheral vision is of lower resolution.
Humans sequentially direct their gaze towards interesting or task-relevant regions of a scene,
integrating information over time to form a coherent understanding of their environment.
This mechanism of selective attention allows humans to efficiently process complex visual scenes under limited computational resources.\\\\
%
In contrast, most artificial vision systems process visual inputs in a uniform manner,
requiring high computational resources to achieve comparable performance to human vision.
They also usually only use a single feedforward pass to process an image, leading to limited interpretability and adaptability to changing environments.\\\\
%
Research in active vision and attention-based reinforcement learning has begun to address these limitations.
Models such as the Recurrent Attention Model (RAM) \citep{mnih2014recurrentmodelsvisualattention} and its variants have demonstrated
the potential of foveated vision and sequential attention mechanisms in artificial agents.

